\documentclass[aip,graphicx,sd, amsmath,amssymb]{revtex4-1}
%\documentclass[aip,reprint]{revtex4-1}

\begin{document}
\section{Importance sampling}
Major Reference: Frenkel\&Smit, Section 3.1.1

\begin{enumerate}
\item[\it{i})] \textbf{[Prepare generating distribution]} \\
1. Evaluate short-time limit of RPMD flux-side correlation function at reactant asymptote $q^{\ddagger} = q^*$
\begin{align}
 \lim_{t \to 0^+} {\tilde{c}_{fs}}(\beta) &= k^{\rm QTST}(q^{\ddagger}, \beta) Q_r(\beta) =  \frac1{(2\pi \hbar)^n}  
\int d\textbf{p}_0 d\textbf{q}_0 \hspace{2pt}  
\rho_{\rm eq} \left(\textbf{p}_0, \textbf{q}_0 \right) \hspace{2pt} 
\times \delta({\bar{q}_0- q^\ddagger}) \frac{\bar{p}_0}{m} \cdot  \hspace{2pt} h(\bar{p}_0), \nonumber \\
\rho_{\rm eq} \left(\textbf{p}_0, \textbf{q}_0 \right) &= e^{-\beta_n H_n^{\rm iso} \left(\textbf{p}_0, \textbf{q}_0 \right)}.
\end{align}
The physical Hamiltonian is a one-dimensional Born-Oppenheimer object 
\begin{align}
\hat{H} = \frac{\hat{p}^2}{2m} + V(\hat{q})
\end{align}
and has a flat reactant asymptote 
\begin{align}
\lim_{q\to -\infty} V(q) = V_0.
\end{align}
The ring-polymer Hamiltonian is
\begin{align}  \label{rpmdH}
H_n^{\rm iso}(\textbf{p}, \textbf{q}) &= \sum_{\alpha=1}^n \frac{p_{\alpha}^2}{2m}  + U_{\text{spr}}(\textbf{q})   + \sum_{\alpha=1}^n V(q_{\alpha})
%\textbf{H}_n^{\rm iso}(\textbf{p}, \textbf{q}) &= \sum_{\alpha=1}^n \frac{p_{\alpha}^2}{2m_n}  + U_{\text{spr}}(\textbf{q})   + \textbf{V}_n^{\rm iso}(\textbf{q})
%\sum_{\alpha=1}^n V(q_{\alpha}), \nonumber \\
\end{align} 
with 
\begin{align}
U_{\text{spr}}(\textbf{q}) &=\frac12 \cdot  \frac{m}{\beta_n^2} \, \sum_{\alpha=1}^n   \left( q_{\alpha} - q_{\alpha+1} \right)^2.
\end{align}

2. Given reactant partition function
\begin{align}
Q_r = \left( \frac{m}{2 \pi \beta} \right)^{1/2} e^{-\beta V_0},
\end{align}
derive $k^{\rm QTST}(q^*, \beta)$.
Similarly, derive the classical limit for QTST rate (with a general dividing surface $q^{\ddagger}$) by utilizing $V(q_\alpha) = V (\bar{q})$.
%Note that 
%\begin{align}
%\int d\textbf{q}_0 \hspace{2pt} \delta(\bar{q}_0 - q^\ddagger )  e^{-\beta_n U_{\text{spr}}(\textbf{q}_0)}
%= \left(\frac{m}{2\pi \beta \hbar^2} \right)^{1/2} \left(\frac{mn}{2\pi \beta \hbar^2} \right)^{-n/2}.
%\end{align}

\item[\it{ii})] \textbf{[Evaluate QTST rate]} \\
Evaluate QTST rate (with a general dividing surface $q^{\ddagger}$) using importance sampling for a symmetrical Eckart barrier model 
\begin{align}
V(q) = {V_0} \left/ {\cosh^2 (q/q_0)}. \right. 
\end{align}
The potential mimics the surface of gas-phase H+$\text{H}_2$ reactive scattering with parameters $m=1061m_e$, $V_0=
%0.425$ eV = $
0.015618462$ Hartree and $q_0=0.734$ Bohr. 

MC algorithm can be expressed as 
\begin{align}
 \langle \mathcal{Q} \rangle_{q^\ddagger} &=
\int dq_1 dq_2 ... dq_n  \hspace{2pt} \delta(\bar{q}_0 - q^\ddagger )  e^{-\beta_n U_{\text{spr}}(\textbf{q}_0)}
 \times \mathcal{Q}(\textbf{q}_0)   \nonumber \\
 &=
\int \sqrt{n} d\bar{q}_0 d\nu_2 ... d\nu_n  \hspace{2pt} \delta(\bar{q}_0 - q^\ddagger )  
\prod_{k=2}^n \exp \left[ -\frac{mn}{2 \beta} \, \lambda_k \nu_k^2 \right]
 \times \mathcal{Q}(\textbf{q}_0)    \nonumber \\
 &=
\int dq_1 dq_2 ... dq_n  \hspace{2pt} \delta(\bar{q}_0 - q^\ddagger ) 
\prod_{k=2}^n \exp \left[ -\frac{mn}{2 \beta} \, \lambda_k \nu_k^2 \right]
 \times \mathcal{Q}(\textbf{q}_0)
\end{align}
in discretized normal mode representation
\begin{align}
& \qquad \hspace{-300pt} \sum_{\alpha=1}^n  (q_\alpha - q_{\alpha+1})^2=\sum_{k=2}^n \lambda_k \nu_k^2, \qquad 
\lambda_k= 2 \left[ 1- \cos \left( \frac{2 \pi \lfloor k/2 \rfloor }n \right) \right] = 4 \sin^2 \left( \frac{\pi \lfloor k/2 \rfloor }n \right),  \nonumber \\
\begin{cases}
&\nu_1=\frac{1}{\sqrt{n}} \sum_{\alpha=1}^n  q_\alpha, \qquad \nu_n=\frac{1}{\sqrt{n}} \sum_{\alpha=1}^n (-1)^\alpha   q_\alpha, \\
&\nu_k=\frac{\sqrt{2}}{\sqrt{n}} \sum_{\alpha=1}^n \sin \left( \frac{2 \pi \alpha \lfloor k/2 \rfloor }n \right) q_\alpha, \qquad k \text{ is even}, \\
&\nu_k=\frac{\sqrt{2}}{\sqrt{n}}  \sum_{\alpha=1}^n \cos \left( \frac{2 \pi \alpha \lfloor k/2 \rfloor }n \right) q_\alpha, \qquad k \text{ is odd}. \\
\end{cases}
\end{align}
where notation $ \lfloor \cdot \rfloor$ represents a floor function. Free ring-polymer distribution can be naturally used as generating distribution.

\item[\it{iii})] \textbf{[Improve statistical estimators]} \\
Now consider calculating the temperature derivative (or potentially mass derivative) of QTST rate. \\
(STILL WORKING HERE)
 
\end{enumerate}

% 
%This section discusses the strategy to calculate stationary temperature $\beta_0$ with the expression
%\begin{align} \label{statbeta}
%E_0 = - \frac1{\tilde{c}_{fs}} \left. \frac{\partial \tilde{c}_{fs}}{\partial \beta} \right|_{\beta_0} , \qquad \beta_0 \in \mathbb{R}.
%\end{align}
%To make RHS of Eq. \ref{statbeta} a purely statistical quantity, reaction rate is approximated as the optimized quantum transition state rate,
%\begin{align}  \label{qtst}
%{\tilde{c}_{fs}}(\beta) \approx &\min_{q^\ddagger} \left[ k_{\rm QTST} (\beta, q^\ddagger) \right]  Q_r(\beta) \nonumber \\
%=& \frac1{(2\pi \hbar)^n}  
%\int d\textbf{p}_0 d\textbf{q}_0 \hspace{2pt}  
%\rho_{\rm eq} \left(\textbf{p}_0, \textbf{q}_0 \right) \hspace{2pt} 
%\times \delta({\bar{q}_0- q^\ddagger}) \frac{\bar{p}_0}{m} \cdot  \hspace{2pt} h(\bar{p}_0).
%\end{align}
%with
%\begin{align}
%\rho_{\rm eq}  \left(\textbf{p}_0, \textbf{q}_0 \right) 
%% = & \hspace{2pt} {\rm tr_e} \left[ e^{-\beta \hat{H}} \right] 
%= & {\rm tr_e} \left[ e^{-\beta_n \textbf{H}_n^{\rm iso} \left(\textbf{p}_0, \textbf{q}_0\right) }  \right] \nonumber \\
%= & \hspace{2pt} e^{-\beta_n \textbf{p}_0^{\rm T} \textbf{p}_0  \left/ 2m \right.} \cdot
%e^{-\beta_n  U_{\text{spr}}(\textbf{q}_0)  } 
%\times \sum_{\gamma} e^{-\beta_n \varepsilon_\gamma^{\rm iso} \left(\textbf{p}_0, \textbf{q}_0\right) }.
%\end{align}
%From now on for conciseness, $q^\ddagger$ will represent the optimized position of dividing surface. We now separate the integration into momentum and position parts,
%\begin{align} \label{pqsep}
%%\langle k \rangle_{t \to 0^+}^{\rm micro} (E_0)  & =\frac1{2\pi i} \int_{\gamma-i\infty}^{\gamma+i\infty} d\beta  \hspace{2pt}  e^{\beta E_0}\hspace{2pt}2\pi \hbar \cdot \tilde{c}_{fs}(t \to 0^+) \nonumber \\
%\tilde{c}_{fs}(\beta) 
%& \approx  \left[  \frac1{(2\pi \hbar)^n}  \int d\textbf{p}_0 \hspace{2pt}  e^{-\beta_n \textbf{p}_0^{\rm T} \textbf{p}_0  \left/ 2m \right.} \cdot  \frac{\bar{p}_0}{m} \hspace{2pt} h({\bar{p}_0}) \right] \nonumber\\
%&\qquad \hspace{100pt} \times 
%\left[ \int d\textbf{q}_0 \hspace{2pt} e^{-\beta_n  U_{\text{spr}}(\textbf{q}_0)  }  \cdot 
%%e^{-\beta_n  \sum_i V(q_i)} 
%\mu(\textbf{q}_0)
%\times \delta({\bar{q}_0- q^\ddagger}) \right] \nonumber \\
%& =  \frac{\sqrt{n}}{2\pi \beta} \left(\frac{ mn}{2\pi \beta} \right)^{(n-1)/2}  \times c_Q
%\end{align}
%with
%\begin{align}
%\mu(\textbf{q}_0) &= \sum_{\gamma} e^{-\beta_n \varepsilon_\gamma^{\rm iso}\left(\textbf{p}_0, \textbf{q}_0\right) } \nonumber \\
%&= {\rm tr_e} \left[ \prod_{\alpha=1}^n e^{-\beta_n \hat{V}(q_\alpha)} \right] 
%=  {\rm tr_e} \left[ e^{-\beta_n \hat{V}(q_1)} e^{-\beta_n \hat{V}(q_2)} \cdots e^{-\beta_n \hat{V}(q_n)} \right].
%\end{align}
%As a result, Eq.~\ref{statbeta} becomes
%\begin{align}
%E_0 = \frac{n+1}{2\beta_0} - \frac1{c_Q} \left. \frac{\partial c_Q}{\partial \beta} \right|_{\beta_0}.
%\end{align}
%The next step is to determine the derivatives of $c_Q$. For general multi-level systems, we have 
%\begin{align}
%\frac{\partial c_Q}{\partial \beta} &=
%\left\langle \frac{mn}{2 \beta^2} \sum_\alpha (q_\alpha-q_{\alpha+1})^2 \cdot \mu(\textbf{q}_0)  
%+ \frac{\partial \mu(\textbf{q}_0)}{\partial \beta} \right\rangle_{ q^\ddagger} , \nonumber \\
%\frac{\partial \mu(\textbf{q}_0)}{\partial \beta} &= -\frac1n  {\rm tr_e} \left[ \hat{V}(q_1)  e^{-\beta_n \hat{V}(q_1)} e^{-\beta_n \hat{V}(q_2)} \cdots e^{-\beta_n \hat{V}(q_n)} \right] \nonumber \\
%&\hspace{60pt} -\frac1n  {\rm tr_e} \left[ e^{-\beta_n \hat{V}(q_1)}   \hat{V}(q_2)  e^{-\beta_n \hat{V}(q_2)} \cdots e^{-\beta_n \hat{V}(q_n)}   \right] \nonumber \\
%&\hspace{90pt} - \cdots -\frac1n  {\rm tr_e} \left[ e^{-\beta_n \hat{V}(q_1)} e^{-\beta_n \hat{V}(q_2)} \cdots  \hat{V}(q_n) e^{-\beta_n \hat{V}(q_n)}   \right] 
%\end{align}
%%and
%%\begin{align}
%% \frac{\partial^2 c_Q}{\partial \beta^2}  &= \left\langle \left[ \frac{mn}{2 \beta^2} \sum_\alpha (q_\alpha-q_{\alpha+1})^2 \right]^2 \cdot \mu(\textbf{q}_0) +  \frac{mn}{ \beta^2} \sum_\alpha (q_\alpha-q_{\alpha+1})^2 \cdot  \frac{\partial \mu(\textbf{q}_0)}{\partial \beta} 
%%- \frac{mn}{ \beta^3} \sum_\alpha (q_\alpha-q_{\alpha+1})^2 \cdot  \frac{\partial^2 \mu(\textbf{q}_0)}{\partial \beta^2}  \right
%%\rangle_{q^\ddagger}
%%\end{align}
%where internal-modes average $\langle \mathcal{Q} \rangle_{q^\ddagger} $ is defined as
%\begin{align}
%\langle \mathcal{Q} \rangle_{q^\ddagger} &\equiv  \int d\textbf{q}_0 \hspace{2pt} \delta(\bar{q}_0 - q^\ddagger )  e^{-\beta_n U_{\text{spr}}(\textbf{q}_0)}
%\times  \mathcal{Q}(\textbf{q}_0).
%\end{align}
%
%It is possible to derive a Virial estimator for the stationary temperature. In other words, Euler's homogeneous function theorem can be applied to get rid of the spring term, i.e.
%\begin{align}
%U_{\text{spr}}(\textbf{q}) 
%&=\frac12 \cdot  \frac{m}{\beta_n^2} \, \sum_{\alpha=1}^n   \left( q_{\alpha} - q_{\alpha+1} \right)^2 
%= \frac12 \sum_{\alpha=1}^n q_\alpha \frac{\partial U_{\text{spr}}(\textbf{q}) }{\partial q_\alpha}.
%\end{align}
%As a result, integration by parts turns out to be
%\begin{align}
%& \left\langle \mu(\textbf{q}_0)  \cdot \frac{mn}{2 \beta^2} \sum_\alpha (q_\alpha-q_{\alpha+1})^2 \right\rangle_{q^\ddagger} \nonumber \\
%=& \frac1{2n} \left\langle \mu(\textbf{q}_0)  \cdot  \sum_{\alpha=1}^n q_\alpha \frac{\partial U_{\text{spr}}(\textbf{q}) }{\partial q_\alpha}   \right\rangle_{q^\ddagger} \nonumber \\
%=&  -\frac1{2 \beta} \int d\textbf{q}_0 \hspace{2pt} \delta(\bar{q}_0 - q^\ddagger ) \mu(\textbf{q}_0)   \cdot  \sum_{\alpha=1}^n q_\alpha \frac{\partial \exp \left[ -\beta_n U_{\text{spr}}(\textbf{q}) \right] }{\partial q_\alpha}  \nonumber \\
%=&  \frac1{2 \beta} \int d\textbf{q}_0 \hspace{2pt} e^{ -\beta_n U_{\text{spr}}(\textbf{q})}
%\cdot  \sum_{\alpha=1}^n \frac{\partial  }{\partial q_\alpha} \left[ q_\alpha \cdot  \delta(\bar{q}_0 - q^\ddagger ) \mu(\textbf{q}_0)    \right] \nonumber \\
%=& \frac1{2 \beta} \int d\textbf{q}_0 \hspace{2pt} e^{ -\beta_n U_{\text{spr}}(\textbf{q})}
%\cdot \left[
%\delta(\bar{q}_0 - q^\ddagger ) \left(n \mu(\textbf{q}_0)  + \sum_\alpha q_\alpha \frac{\partial \mu(\textbf{q}_0)}{\partial q_\alpha}  \right)   
%+   \mu(\textbf{q}_0)  \sum_\alpha q_\alpha \frac{\partial  \delta(\bar{q}_0 - q^\ddagger ) }{\partial  q_\alpha} \right]
%\end{align}
%The derivative of delta function is addressed by substituting in the relation
%\begin{align}
%\frac{\partial  \delta(\bar{q}_0 - q^\ddagger ) }{\partial  q_\alpha} 
%= \frac{d \delta(\bar{q}_0 - q^\ddagger )}{d \bar{q}_0} \frac{\partial   \bar{q}_0 }{\partial  q_\alpha} 
%= \frac1n \frac{d \delta(\bar{q}_0 - q^\ddagger )}{d \bar{q}_0}
%\end{align}
%followed by integration by parts
%\begin{align}
% &\int d\textbf{q}_0 \hspace{2pt} e^{ -\beta_n U_{\text{spr}}(\textbf{q})}  \cdot  \mu(\textbf{q}_0) \sum_\alpha q_\alpha \frac{\partial  \delta(\bar{q}_0 - q^\ddagger ) }{\partial  q_\alpha}  \nonumber \\
%= &\int d\textbf{q}_0 \hspace{2pt} e^{ -\beta_n U_{\text{spr}}(\textbf{q})}  \cdot \mu(\textbf{q}_0)  \cdot  { \bar{q}_0} \frac{d \delta(\bar{q}_0 - q^\ddagger )}{d \bar{q}_0}  \nonumber \\
%=& - \left\langle \left[ \mu(\textbf{q}_0)  +  { \bar{q}_0} \sum_\alpha \frac{\partial \mu(\textbf{q}_0)}{\partial q_\alpha} \right]  \delta(\bar{q}_0 - q^\ddagger ) \right\rangle_{q^\ddagger}.
%\end{align}
%Finally we have the estimator
%\begin{align}  \label{stattempfl}
%E = \frac{1}{\beta} - \left. { \left\langle 
%\frac1{2\beta} \sum_\alpha \left( q_\alpha - \bar{q}_0 \right) \frac{\partial  \mu(\textbf{q}_0)}{\partial q_\alpha} + \frac{\partial \mu(\textbf{q}_0)}{\partial \beta}  \right\rangle_{q^\ddagger}} \right/ 
%\left\langle \mu(\textbf{q}_0)  \right\rangle_{q^\ddagger}.
%\end{align}
%Especially at the adiabatic limit since $\mu=e^{-\beta_n  \sum_\alpha V(q_\alpha)}$, the last expression reduces to
%%For single-level systems  we have the primitive estimator
%%\begin{align} \label{pri1l}
%%\frac{\partial c_Q}{\partial \beta} &= \int d\textbf{q}_0 e^{- m \sum_\alpha (q_\alpha-q_{\alpha+1})^2 /2 \beta_n} \cdot 
%%e^{-\beta_n  \sum_\alpha V(q_\alpha)} \times \delta({\bar{q}_0- q^\ddagger}) \left[ \frac{mn}{2 \beta^2} \sum_\alpha (q_\alpha-q_{\alpha+1})^2 - \frac1n \sum_\alpha V(q_\alpha)\right] \nonumber \\
%% &= \left\langle e^{-\beta_n  \sum_\alpha V(q_\alpha)} \cdot \left[ \frac{mn}{2 \beta^2} \sum_\alpha (q_\alpha-q_{\alpha+1})^2 - \frac1n \sum_\alpha V(q_\alpha) \right] \right\rangle_{q^\ddagger}
%%\end{align}
%%and similarly the second derivative
%%\begin{align}
%%\frac{\partial^2 \ln \hspace{2pt} \tilde{c}_{fs} }{\partial \beta^2 } &= 
%%\left\langle e^{-\beta_n  \sum_\alpha V(q_\alpha)} \cdot \left\{
%%\left[ \frac{mn}{2 \beta^2} \sum_\alpha (q_\alpha-q_{\alpha+1})^2 - \frac1n \sum_\alpha V(q_\alpha)\right]^2 - \frac{mn}{ \beta^3} \sum_\alpha (q_\alpha-q_{\alpha+1})^2 \right\}
%%\right\rangle_{ q^\ddagger}.
%%\end{align}
%\begin{align} \label{stattemp1l}
%E = \frac{1}{\beta} + \left. { \left\langle e^{-\beta_n  \sum_\alpha V(q_\alpha)} \cdot  
%\left[ \frac1{2n} \sum_\alpha \left( q_\alpha - \bar{q}_0 \right) \frac{\partial V(q_\alpha)}{\partial q_\alpha} + \frac1n \sum_\alpha V(q_\alpha) \right] \right\rangle_{q^\ddagger}} \right/ 
%\left\langle e^{-\beta_n  \sum_\alpha V(q_\alpha)}  \right\rangle_{q^\ddagger}.
%\end{align}
%
%An attracting feature of Eq. \ref{stattempfl} is that it is connected to the result from classical transition state theory if the ring-polymer is forced to be contracted at high temperature. For example in one-level systems, Eq. \ref{stattemp1l} becomes
%\begin{align}
%\lim_{\rm classical} E_0 = \frac{1}{\beta_0} + V(q^\ddagger),
%\end{align}
%and recovers the result from 
%\begin{align}
%{\tilde{c}_{fs}^{\rm CTST}} = 1 / \left( 2 \pi \beta \hbar \right) \cdot \exp \left[ -\beta V(q^\ddagger) \right] 
%\end{align}
%and also the classical approximation gives
%\begin{align}
%\frac{\partial^2 \ln \hspace{2pt} {\tilde{c}_{fs}^{\rm CTST}}  }{\partial \beta^2 } = \frac1{\beta^2}.
%\end{align}
%Similarly for multi-level conditions, 
%\begin{align}
%{\tilde{c}_{fs}^{\rm CTST}} = \frac1{ 2 \pi \beta \hbar} \cdot 
%\sum_\gamma e^{ -\beta V_\gamma (q^\ddagger)},
%\end{align}
%with $\gamma$ the index for adiabatic states and it leads to the relation
%\begin{align}
%E_0 = - \frac{\partial \ln \hspace{2pt} {\tilde{c}_{fs}^{\rm CTST}}  }{\partial \beta} &= \frac1\beta + 
%\frac{\sum_\gamma V_\gamma \cdot e^{ -\beta V_\gamma (q^\ddagger)} }{\sum_\gamma e^{ -\beta V_\gamma (q^\ddagger)}}
%\approx  \frac1\beta + V_{\rm min}(q^\ddagger),
%\end{align}
%which is the classical asymptotic expression of Eq. \ref{stattempfl}.


\end{document}
